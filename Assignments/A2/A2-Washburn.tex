%This is my super simple Real Analysis Homework template

\documentclass{AssignmentCUNY}
\usepackage{amsmath}
\usepackage[]{amsthm} %lets us use \begin{proof}
\usepackage[]{amssymb} %gives us the character \varnothing
\usepackage[utf8]{inputenc}
\usepackage{enumitem}
\usepackage{mathrsfs}
\usepackage{nicematrix}
\usepackage{xifthen}

\setlength{\abovedisplayskip}{1pt}
\setlength{\belowdisplayskip}{1pt}
\setlength{\abovedisplayshortskip}{1pt}
\setlength{\belowdisplayshortskip}{1pt}

\setcounter{MaxMatrixCols}{32}

\renewcommand{\thefootnote}{\fnsymbol{footnote}}


\newcommand{\Det}[1]{\ensuremath{\text{det}\left(#1\right)}}
\newcommand{\PropSep}{\quad\!\!}
\newcommand{\Rnn}{\ensuremath{\mathbb{R}_{n \times n}}}
\newcommand{\Sn}{\ensuremath{\mathcal{S}_{n}}}
\newcommand{\VD}{\Vdots}
\newcommand{\R}[1]{\rotate~#1}
\newcommand{\C}[1]{#1\hspace*{1mm}}

\newcommand{\MCellI}[4]{\ensuremath{\sum\limits_{i=1}^{n} #1_{#3,i}*#2_{i,#4}}}%
\newcommand{\MCellJ}[4]{\ensuremath{\sum\limits_{j=1}^{n} #1_{#3,j}*#2_{j,#4}}}%
\newcommand{\MCellIJ}[5]{\ensuremath{\sum\limits_{i=1}^{n}(#1_{#4,i}*(\sum\limits_{j=1}^{n} #2_{i,j}*#3_{j,#5}))}}%
\newcommand{\MCellIJD}[5]{\ensuremath{\sum\limits_{i=1}^{n}\sum\limits_{j=1}^{n} #1_{#4,i}*#2_{i,j}*#3_{j,#5}}}%
\newcommand{\MCellJI}[5]{\ensuremath{\sum\limits_{j=1}^{n}((\sum\limits_{j=1}^{n} #1_{#4,j}*#2_{i,j})*#3_{#5,j})}}%

%\newcommand{cmd}[args][default]{def}

%
\newcommand{\BO}{\ensuremath{\mathbf{1}}}
%

\newcommand{\ClosureQuantification}[1]{\ensuremath{\forall a, b \in #1}}
\newcommand{\ClosureProperty}[2]{\ensuremath{a #1 b \in #2}}
		
\newcommand{\AssociativityQuantification}[1]{\ensuremath{\forall a, b, c \in #1}}
\newcommand{\AssociativityProperty}[1]{\ensuremath{a #1 (b #1 c) = (a #1 b) #1 c}}

\newcommand{\IdentityQuantification}[1]{\ensuremath{\exists e \in #1\; \forall a \in #1}}
\newcommand{\IdentityProperty}[1]{\ensuremath{e #1 a = a = a #1 e}}

\newcommand{\InverseQuantification}[1]{\ensuremath{\forall a \in #1\; \exists a^{'} \in #1}}
\newcommand{\InverseProperty}[1]{\ensuremath{a #1 a^{'} = e = a^{'} #1 a}}


\AssignmentNumber{2}%
\CourseName{Coding Theory}
\CourseNumber{85020}
\StudentName{Alex Washburn}

\begin{document}
\CoverPage
\Problem{%
Show that non-singular $n \times n$ real matrices with matrix multiplication is a group.%
}%

Let $\Rnn$ be the set of (non-singular) $n \times n$ real matrices.

Let us define the binary operator for matrix multiplication as $\left(\ast\right) \colon \Rnn \times \Rnn \to \Rnn$.

We must show that the \emph{four} properties of an algebraic group hold for $\mathbb{R}_{n \times n}$ under $\left(\ast\right)$.\\


\begingroup
\setlength{\tabcolsep}{10pt} % Default value: 6pt
\renewcommand{\arraystretch}{1.5} % Default value: 1
$\mathbb{R}_{n \times n}$ forms an algebraic group under $\left(\ast\right)$ if an only if:\\
\[
\begin{array}[t]{c|lrl}%
\text{1} & \text{Closure:}       &       \ClosureQuantification{\Rnn} &       \ClosureProperty{\ast}{\Rnn} \\
\text{2} & \text{Associativity:} & \AssociativityQuantification{\Rnn} & \AssociativityProperty{\ast} \\
\text{3} & \text{Identity:}      &      \IdentityQuantification{\Rnn} &      \IdentityProperty{\ast} \\
\text{4} & \text{Inverse:}       &       \InverseQuantification{\Rnn} &       \InverseProperty{\ast}
\end{array}
\]
\endgroup\\


\pagebreak
\SubProblem{Closure: $\quad\ClosureQuantification{\Rnn} \quad\quad \ClosureProperty{\ast}{\Rnn}$}

Consider the definition of $\left( \ast \right)$, which takes two matrices $a$ and $b$ with dimensions $x \times y$ and $y \times z$ (respectively) and produces a matrix $(a \ast b)$ with dimensions $x \times z$. Note that when $a, b \in \Rnn$ then $x = y = z = n$. Therefore, $\ClosureQuantification{\Rnn} \PropSep\ClosureProperty{\ast}{\Rnn}$

	
\SubProblem{Associativity: $\quad\AssociativityQuantification{\Rnn} \quad\quad \AssociativityProperty{\ast}$}

Associativity of $\Rnn$ under $\left( \ast \right)$ ultimately follows from the associativity of the real numbers $\mathbb{R}$.
Observe the following equalities \emph{(next page)}:


\begin{align*}
a \ast (b \ast c)~&=~a \ast \begin{pNiceMatrix}%
[margin,nullify-dots]
\MCellI{b}{c}{1}{1} & \MCellI{b}{c}{1}{2} & \Cdots & \MCellI{b}{c}{1}{n} \\
\MCellI{b}{c}{2}{1} & \MCellI{b}{c}{2}{2} & \Cdots & \MCellI{b}{c}{2}{n} \\
\VD                 & \VD                 & \Ddots & \VD                \\
\MCellI{b}{c}{n}{1} & \MCellI{b}{c}{n}{2} & \Cdots & \MCellI{b}{c}{n}{n} \\
\end{pNiceMatrix}\\
&=~\begin{pNiceMatrix}%
[margin,nullify-dots]
\MCellIJ{a}{b}{c}{1}{1} & \MCellIJ{a}{b}{c}{1}{2} & \Cdots & \MCellIJ{a}{b}{c}{1}{n} \\
\MCellIJ{a}{b}{c}{2}{1} & \MCellIJ{a}{b}{c}{2}{2} & \Cdots & \MCellIJ{a}{b}{c}{2}{n} \\
\VD                & \VD                & \Ddots & \VD                \\
\MCellIJ{a}{b}{c}{n}{1} & \MCellIJ{a}{b}{c}{n}{2} & \Cdots & \MCellIJ{a}{b}{c}{n}{n} \\
\end{pNiceMatrix}\\
&=~\begin{pNiceMatrix}%
[margin,nullify-dots]
\MCellIJD{a}{b}{c}{1}{1} & \MCellIJD{a}{b}{c}{1}{2} & \Cdots & \MCellIJD{a}{b}{c}{1}{n} \\
\MCellIJD{a}{b}{c}{2}{1} & \MCellIJD{a}{b}{c}{2}{2} & \Cdots & \MCellIJD{a}{b}{c}{2}{n} \\
\VD                & \VD                & \Ddots & \VD                \\
\MCellIJD{a}{b}{c}{n}{1} & \MCellIJD{a}{b}{c}{n}{2} & \Cdots & \MCellIJD{a}{b}{c}{n}{n} \\
\end{pNiceMatrix}\\
&=~\begin{pNiceMatrix}%
[margin,nullify-dots]
\MCellJI{a}{b}{c}{1}{1} & \MCellJI{a}{b}{c}{1}{2} & \Cdots & \MCellJI{a}{b}{c}{1}{n} \\
\MCellJI{a}{b}{c}{2}{1} & \MCellJI{a}{b}{c}{2}{2} & \Cdots & \MCellJI{a}{b}{c}{2}{n} \\
\VD                      & \VD                      & \Ddots & \VD                      \\
\MCellJI{a}{b}{c}{n}{1} & \MCellJI{a}{b}{c}{n}{2} & \Cdots & \MCellJI{a}{b}{c}{n}{n} \\
\end{pNiceMatrix}\\
&=~\begin{pNiceMatrix}%
[margin,nullify-dots]
\MCellJ{a}{b}{1}{1} & \MCellJ{a}{b}{1}{2} & \Cdots & \MCellJ{a}{b}{1}{n} \\
\MCellJ{a}{b}{2}{1} & \MCellJ{a}{b}{2}{2} & \Cdots & \MCellJ{a}{b}{2}{n} \\
\VD                 & \VD                 & \Ddots & \VD                \\
\MCellJ{a}{b}{n}{1} & \MCellJ{a}{b}{n}{2} & \Cdots & \MCellJ{a}{b}{n}{n} \\
\end{pNiceMatrix} \ast c \\
&=~(a \ast b) \ast c \\
\end{align*}

	
\pagebreak
\SubProblem{Identity: $\quad\IdentityQuantification{\Rnn} \quad\quad \IdentityProperty{\ast}$}

Consider the identity element:

\[
e \quad=\quad I_{n \times n} \quad=\quad \begin{pNiceMatrix}%
[margin,name=RepetitionCode,first-row,first-col,nullify-dots]
~   & \R 1& \R 2& \R 3& \Cdots & \R n\\
1   & \BO &  0  &  0  & \Cdots &  0  \\
2   &  0  & \BO &  0  & \Cdots &  0  \\
3   &  0  &  0  & \BO & \Cdots &  0  \\
\VD & \VD & \VD & \VD & \Ddots & \VD \\
n   &  0  &  0  &  0  & \Cdots & \BO \\
\end{pNiceMatrix}
\]

Observe the following equalities:
\newcommand{\ICellR}[3]{\ensuremath{#1_{#2,#3}*1 + \sum\limits_{i\in[1,n] \setminus {#2}} #1_{#2,i}*0}}%
\newcommand{\ICellC}[3]{\ensuremath{1*#1_{#2,#3} + \sum\limits_{i\in[1,n] \setminus {#2}} 0*#1_{#2,i}}}%

\begin{align*}
e \ast a~&=~\begin{pNiceMatrix}%
[margin,nullify-dots]
\BO &  0  &  0  & \Cdots &  0  \\
0   & \BO &  0  & \Cdots &  0  \\
0   &  0  & \BO & \Cdots &  0  \\
\VD & \VD & \VD & \Ddots & \VD \\
0   &  0  &  0  & \Cdots & \BO \\
\end{pNiceMatrix} \ast \begin{pNiceMatrix}%
[margin,nullify-dots]
a_{1,1} & a_{1,2} & a_{1,3} & \Cdots & a_{n,1} \\
a_{2,1} & a_{2,2} & a_{2,3} & \Cdots & a_{n,2} \\
a_{3,1} & a_{3,2} & a_{3,3} & \Cdots & a_{n,3} \\
\VD     & \VD     & \VD     & \Ddots & \VD     \\
a_{n,1} & a_{n,2} & a_{n,3} & \Cdots & a_{n,n} \\
\end{pNiceMatrix}\\
&=~\begin{pNiceMatrix}%
[margin,nullify-dots]
\ICellC{a}{1}{1} & \ICellC{a}{1}{2} & \Cdots & \ICellC{a}{1}{n} \\
\ICellC{a}{2}{1} & \ICellC{a}{2}{2} & \Cdots & \ICellC{a}{2}{n} \\
\VD              & \VD              & \Ddots & \VD              \\
\ICellC{a}{n}{1} & \ICellC{a}{n}{2} & \Cdots & \ICellC{a}{n}{n} \\
\end{pNiceMatrix}\\
&=~\begin{pNiceMatrix}%
[margin,nullify-dots]
a_{1,1} & a_{1,2} & a_{1,3} & \Cdots & a_{n,1} \\
a_{2,1} & a_{2,2} & a_{2,3} & \Cdots & a_{n,2} \\
a_{3,1} & a_{3,2} & a_{3,3} & \Cdots & a_{n,3} \\
\VD     & \VD     & \VD     & \Ddots & \VD     \\
a_{n,1} & a_{n,2} & a_{n,3} & \Cdots & a_{n,n} \\
\end{pNiceMatrix}\\
&=~a \\
&=~\begin{pNiceMatrix}%
[margin,nullify-dots]
a_{1,1} & a_{1,2} & a_{1,3} & \Cdots & a_{n,1} \\
a_{2,1} & a_{2,2} & a_{2,3} & \Cdots & a_{n,2} \\
a_{3,1} & a_{3,2} & a_{3,3} & \Cdots & a_{n,3} \\
\VD     & \VD     & \VD     & \Ddots & \VD     \\
a_{n,1} & a_{n,2} & a_{n,3} & \Cdots & a_{n,n} \\
\end{pNiceMatrix}\\
&=~\begin{pNiceMatrix}%
[margin,nullify-dots]
\ICellR{a}{1}{1} & \ICellR{a}{1}{2} & \Cdots & \ICellR{a}{1}{n} \\
\ICellR{a}{2}{1} & \ICellR{a}{2}{2} & \Cdots & \ICellR{a}{2}{n} \\
\VD              & \VD              & \Ddots & \VD              \\
\ICellR{a}{n}{1} & \ICellR{a}{n}{2} & \Cdots & \ICellR{a}{n}{n} \\
\end{pNiceMatrix}\\
&=~\begin{pNiceMatrix}%
[margin,nullify-dots]
\BO &  0  &  0  & \Cdots &  0  \\
0   & \BO &  0  & \Cdots &  0  \\
0   &  0  & \BO & \Cdots &  0  \\
\VD & \VD & \VD & \Ddots & \VD \\
0   &  0  &  0  & \Cdots & \BO \\
\end{pNiceMatrix}\ast \begin{pNiceMatrix}%
[margin,nullify-dots]
a_{1,1} & a_{1,2} & a_{1,3} & \Cdots & a_{n,1} \\
a_{2,1} & a_{2,2} & a_{2,3} & \Cdots & a_{n,2} \\
a_{3,1} & a_{3,2} & a_{3,3} & \Cdots & a_{n,3} \\
\VD     & \VD     & \VD     & \Ddots & \VD     \\
a_{n,1} & a_{n,2} & a_{n,3} & \Cdots & a_{n,n} \\
\end{pNiceMatrix}\\
&=~a \ast e 
\end{align*}


\pagebreak
\SubProblem{Inverse: $\quad\InverseQuantification{\Rnn} \quad\quad \InverseProperty{\ast}$}

Let $M_{i,j}$ be the ``minor'' of $a$,  an $(n-1) \times (n-1)$ matrix which is a copy of $a$ with row $i$ and column $j$ removed. 

Let $\Det{x}$ be the determinant of $x$ where:

\[
\Det{x_{m \times m}}~=~\begin{cases}
	a_{1,1} & \text{if } m=1\\
	(a_{1,1}*a_{2,2}) - (a_{1,2}*a_{2,1}) & \text{if } m=2\\
	\sum\limits_{i=1}^{m}\sum\limits_{j=1}^{m} (-1)^{i+j}a_{i,j}*\Det{M_{i,j}} & \text{otherwise }
\end{cases}
\]

By definition, all $a \in \Rnn$ is non-singular.
Hence, the determinant is non-zero and can be used to construct an inverse $a^{'}$ for each $a \in \Rnn$.
Since the $\Det{x}$ is defined recursively by cases, we must show the construction of $a^{'}$ for each case.\\


\textbf{Case 1:} Suppose that $n = 1$.

Then all $a \in \mathbb{R}_{1 \times 1}$ has an inverse of the following form:

\[
a^{'}~=~\begin{pNiceMatrix}%
	[margin,nullify-dots]
	\frac{1}{a_{1,1}} \\
\end{pNiceMatrix}
\]

Because $a$ is non-singular, the following equalities hold:

\begin{align*}
a \ast a^{'}~&=~\begin{pNiceMatrix}%
[margin,nullify-dots]
a_{1,1} \\
\end{pNiceMatrix}\ast \begin{pNiceMatrix}%
[margin,nullify-dots]
\frac{1}{a_{1,1}} \\
\end{pNiceMatrix} \\
&=~\begin{pNiceMatrix}%
[margin,nullify-dots]
\BO \\
\end{pNiceMatrix}\\
&=~e \\
&=~\begin{pNiceMatrix}%
[margin,nullify-dots]
\BO \\
\end{pNiceMatrix}\\
&=~\begin{pNiceMatrix}%
[margin,nullify-dots]
\frac{1}{a_{1,1}} \\
\end{pNiceMatrix}\ast \begin{pNiceMatrix}%
[margin,nullify-dots]
a_{1,1} \\
\end{pNiceMatrix} \\&=~a^{'} \ast a
\end{align*}\\



\textbf{Case 2:} Suppose that $n = 2$.

Then all $a \in \mathbb{R}_{2 \times 2}$ has an inverse of the following form:

\[
a^{'}~=~\begin{pNiceMatrix}%
[margin,nullify-dots]
\frac{a_{2,2}}{a_{1,1}*a_{2,2} - a_{1,2}*a_{2,1}} & \frac{-1*a_{1,2}}{a_{1,1}*a_{2,2} - a_{1,2}*a_{2,1}} \\
\frac{-1*a_{2,1}}{a_{1,1}*a_{2,2} - a_{1,2}*a_{2,1}} & \frac{a_{1,1}}{a_{1,1}*a_{2,2} - a_{1,2}*a_{2,1}} \\
\end{pNiceMatrix}
\]\\

Because $a$ is non-singular, the following equalities hold \emph{(next page)}:

\begin{align*}
a \ast a^{'}~&=~\begin{pNiceMatrix}%
[margin,nullify-dots]
a_{1,1} & a_{1,2} \\
a_{2,1} & a_{2,2} \\
\end{pNiceMatrix} \ast \begin{pNiceMatrix}%
[margin,nullify-dots]
\frac{a_{2,2}}{a_{1,1}*a_{2,2} - a_{1,2}*a_{2,1}} & \frac{-1*a_{1,2}}{a_{1,1}*a_{2,2} - a_{1,2}*a_{2,1}} \\
\frac{-1*a_{2,1}}{a_{1,1}*a_{2,2} - a_{1,2}*a_{2,1}} & \frac{a_{1,1}}{a_{1,1}*a_{2,2} - a_{1,2}*a_{2,1}} \\
\end{pNiceMatrix} \\
&=~\begin{pNiceMatrix}%
[margin,nullify-dots]
\frac{a_{1,1}*a_{2,2} - a_{1,2}*a_{2,1}}{a_{1,1}*a_{2,2} - a_{1,2}*a_{2,1}} & \frac{a_{1,2}*a_{1,1} - a_{1,1}*a_{1,2}}{a_{1,1}*a_{2,2} - a_{1,2}*a_{2,1}} \\
\frac{a_{1,1}*a_{1,2} - a_{1,2}*a_{1,1}}{a_{1,1}*a_{2,2} - a_{1,2}*a_{2,1}} & \frac{a_{2,2}*a_{1,1} - a_{2,1}*a_{1,2}}{a_{1,1}*a_{2,2} - a_{1,2}*a_{2,1}} \\\end{pNiceMatrix} \\
&=~\begin{pNiceMatrix}%
[margin,nullify-dots]
\BO &  0  \\
0   & \BO \\
\end{pNiceMatrix}\\
&=~e \\
&=~\begin{pNiceMatrix}%
[margin,nullify-dots]
\BO &  0  \\
0   & \BO \\
\end{pNiceMatrix}\\
&=~\begin{pNiceMatrix}%
[margin,nullify-dots]
\frac{a_{1,1}*a_{2,2} - a_{1,2}*a_{2,1}}{a_{1,1}*a_{2,2} - a_{1,2}*a_{2,1}} & \frac{a_{1,2}*a_{1,1} - a_{1,1}*a_{1,2}}{a_{1,1}*a_{2,2} - a_{1,2}*a_{2,1}} \\
\frac{a_{1,1}*a_{1,2} - a_{1,2}*a_{1,1}}{a_{1,1}*a_{2,2} - a_{1,2}*a_{2,1}} & \frac{a_{2,2}*a_{1,1} - a_{2,1}*a_{1,2}}{a_{1,1}*a_{2,2} - a_{1,2}*a_{2,1}} \\
\end{pNiceMatrix}\\
&=~\begin{pNiceMatrix}%
[margin,nullify-dots]
\frac{a_{2,2}}{a_{1,1}*a_{2,2} - a_{1,2}*a_{2,1}} & \frac{-1*a_{1,2}}{a_{1,1}*a_{2,2} - a_{1,2}*a_{2,1}} \\
\frac{-1*a_{2,1}}{a_{1,1}*a_{2,2} - a_{1,2}*a_{2,1}} & \frac{a_{1,1}}{a_{1,1}*a_{2,2} - a_{1,2}*a_{2,1}} \\
\end{pNiceMatrix} \ast \begin{pNiceMatrix}%
[margin,nullify-dots]
a_{1,1} & a_{1,2} \\
a_{2,1} & a_{2,2} \\
\end{pNiceMatrix} \\
&=~a^{'} \ast a
\end{align*}\\


\newcommand{\CF}[2]{\ensuremath{\frac{(-1)^{#1+#2} * \Det{M_{#2,#1}}}{\Det{a}}}}
\newcommand{\IV}[2]{\ensuremath{\sum\limits_{i=1} \frac{(-1)^{i+#2} * a_{#1,i} * \Det{M_{#2,i}}}{\Det{a}}}}

\textbf{Case 3:} Suppose that $n \ge 2$.

Then all $a \in \Rnn$ has an inverse of the following form:

\[
a^{'}~=~\begin{pNiceMatrix}%
[margin,nullify-dots]
\CF{1}{1} & \CF{1}{2} & \CF{1}{3} & \Cdots & \CF{n}{1} \\
\CF{2}{1} & \CF{2}{2} & \CF{2}{3} & \Cdots & \CF{n}{2} \\
\CF{3}{1} & \CF{3}{2} & \CF{3}{3} & \Cdots & \CF{n}{3} \\
\VD       & \VD       & \VD       & \Ddots & \VD       \\
\CF{n}{1} & \CF{n}{2} & \CF{n}{3} & \Cdots & \CF{n}{n} \\
\end{pNiceMatrix}
\]

Because $a$ is non-singular, the following equalities hold \emph{(next page)}:

\begin{align*}
a \ast a^{'}~&=~\begin{pNiceMatrix}%
[margin,nullify-dots]
a_{1,1} & a_{1,2} & a_{1,3} & \Cdots & a_{n,1} \\
a_{2,1} & a_{2,2} & a_{2,3} & \Cdots & a_{n,2} \\
a_{3,1} & a_{3,2} & a_{3,3} & \Cdots & a_{n,3} \\
\VD     & \VD     & \VD     & \Ddots & \VD     \\
a_{n,1} & a_{n,2} & a_{n,3} & \Cdots & a_{n,n} \\
\end{pNiceMatrix} \ast \begin{pNiceMatrix}%
[margin,nullify-dots]
\CF{1}{1} & \CF{1}{2} & \CF{1}{3} & \Cdots & \CF{n}{1} \\
\CF{2}{1} & \CF{2}{2} & \CF{2}{3} & \Cdots & \CF{n}{2} \\
\CF{3}{1} & \CF{3}{2} & \CF{3}{3} & \Cdots & \CF{n}{3} \\
\VD       & \VD       & \VD       & \Ddots & \VD       \\
\CF{n}{1} & \CF{n}{2} & \CF{n}{3} & \Cdots & \CF{n}{n} \\
\end{pNiceMatrix}\\
&=~\begin{pNiceMatrix}%
[margin,nullify-dots]
\IV{1}{1} & \IV{1}{2} & \IV{1}{3} & \Cdots & \IV{n}{1} \\
\IV{2}{1} & \IV{2}{2} & \IV{2}{3} & \Cdots & \IV{n}{2} \\
\IV{3}{1} & \IV{3}{2} & \IV{3}{3} & \Cdots & \IV{n}{3} \\
\VD       & \VD       & \VD       & \Ddots & \VD       \\
\IV{n}{1} & \IV{n}{2} & \IV{n}{3} & \Cdots & \IV{n}{n} \\
\end{pNiceMatrix}\\
&=~\begin{pNiceMatrix}%
[margin,nullify-dots]
\BO &  0  &  0  & \Cdots &  0  \\
0   & \BO &  0  & \Cdots &  0  \\
0   &  0  & \BO & \Cdots &  0  \\
\VD & \VD & \VD & \Ddots & \VD \\
0   &  0  &  0  & \Cdots & \BO \\
\end{pNiceMatrix}\\
&=~e
&=~\begin{pNiceMatrix}%
[margin,nullify-dots]
\BO &  0  &  0  & \Cdots &  0  \\
0   & \BO &  0  & \Cdots &  0  \\
0   &  0  & \BO & \Cdots &  0  \\
\VD & \VD & \VD & \Ddots & \VD \\
0   &  0  &  0  & \Cdots & \BO \\
\end{pNiceMatrix}\\
&=~\begin{pNiceMatrix}%
[margin,nullify-dots]
\IV{1}{1} & \IV{1}{2} & \IV{1}{3} & \Cdots & \IV{n}{1} \\
\IV{2}{1} & \IV{2}{2} & \IV{2}{3} & \Cdots & \IV{n}{2} \\
\IV{3}{1} & \IV{3}{2} & \IV{3}{3} & \Cdots & \IV{n}{3} \\
\VD       & \VD       & \VD       & \Ddots & \VD       \\
\IV{n}{1} & \IV{n}{2} & \IV{n}{3} & \Cdots & \IV{n}{n} \\
\end{pNiceMatrix}\\
&=~\begin{pNiceMatrix}%
[margin,nullify-dots]
\CF{1}{1} & \CF{1}{2} & \CF{1}{3} & \Cdots & \CF{n}{1} \\
\CF{2}{1} & \CF{2}{2} & \CF{2}{3} & \Cdots & \CF{n}{2} \\
\CF{3}{1} & \CF{3}{2} & \CF{3}{3} & \Cdots & \CF{n}{3} \\
\VD       & \VD       & \VD       & \Ddots & \VD       \\
\CF{n}{1} & \CF{n}{2} & \CF{n}{3} & \Cdots & \CF{n}{n} \\
\end{pNiceMatrix} \ast \begin{pNiceMatrix}%
[margin,nullify-dots]
a_{1,1} & a_{1,2} & a_{1,3} & \Cdots & a_{n,1} \\
a_{2,1} & a_{2,2} & a_{2,3} & \Cdots & a_{n,2} \\
a_{3,1} & a_{3,2} & a_{3,3} & \Cdots & a_{n,3} \\
\VD     & \VD     & \VD     & \Ddots & \VD     \\
a_{n,1} & a_{n,2} & a_{n,3} & \Cdots & a_{n,n} \\
\end{pNiceMatrix}\\
&=~a^{'} \ast a \\
\end{align*}



\Problem{%
Given two permutations $\bigl[ 0 \mapsto 1,\, 1 \mapsto 2,\, 2 \mapsto 0 \bigr]$ and $\bigl[ 0 \mapsto 1,\, 1 \mapsto 0,\, 2 \mapsto 2 \bigr]$, their composition is another permutation $\bigl[ 0 \mapsto 0,\, 1 \mapsto 2,\, 2 \mapsto 1 \bigr]$.\\[5mm]
Show that permutations over $n$ elements with composition is a group.
}%

Let $\Sn$ be the set of permutations on $n$ elements.

Let us define the binary operator for permutation composition as $\left(\ast\right) \colon \Sn \times \Sn \to \Sn$.

We must show that the \emph{four} properties of an algebraic group hold for $\Sn$ under $\left(\ast\right)$.\\


\begingroup
\setlength{\tabcolsep}{10pt} % Default value: 6pt
\renewcommand{\arraystretch}{1.5} % Default value: 1
$\Sn$ forms an algebraic group under $\left(\ast\right)$ if an only if:\\
\[
\begin{array}[t]{c|lrl}%
	\text{1} & \text{Closure:}       &       \ClosureQuantification{\Sn} &       \ClosureProperty{\circ}{\Sn} \\
	\text{2} & \text{Associativity:} & \AssociativityQuantification{\Sn} & \AssociativityProperty{\circ} \\
	\text{3} & \text{Identity:}      &      \IdentityQuantification{\Sn} &      \IdentityProperty{\circ} \\
	\text{4} & \text{Inverse:}       &       \InverseQuantification{\Sn} &       \InverseProperty{\circ}
\end{array}
\]
\endgroup\\

\newcommand{\Nn}{\ensuremath{\mathbb{N}_{n}}}
\newcommand{\Homo}[1]{\ensuremath{\textsc{t}\left(#1\right)}}
\newcommand{\HomoI}[1]{\ensuremath{\textsc{t}^{'}\left(#1\right)}}

First, we will establish a key relationship:

Let $\Nn = \left\{ x : x \in \mathbb{N} \text{ and } x < n\right\} \subset \mathbb{N}$.
Let $\textsc{t}\colon \Sn \to \left(\Nn \to \Nn\right)$ be the bijective function uniquely mapping an element of $\Sn$ and to a bijective function mapping elements from $\Nn$ onto itself.
Note $\Sn$ and $\left\{ \Homo{x} : x \in \Sn \right\}$ are homomorphic with $\textsc{t}$ and its inverse $\textsc{t}^{'}$ mapping between the two sets.


\SubProblem{Closure: $\quad\ClosureQuantification{\Sn}\quad\quad\ClosureProperty{\ast}{\Sn}$}

Let $a$ and $b$ be permutations from $\Sn$.
We will use the properties of function composition $\left(\circ\right)$ to derive that $a \ast b$ is closed. 
This can be illustrated via lambda calculus:

\begin{align*}
a \ast b~&=~\HomoI{\Homo{a \ast b}} \\
&=~\HomoI{\Homo{a} \circ \Homo{b}} \\
&=~\HomoI{(\lambda_{1}\colon \Nn \to \Nn) \circ (\lambda_{2}\colon \Nn \to \Nn)} \\
&=~\HomoI{\lambda_{3}\colon \Nn \to \Nn} \\
\end{align*}

We know that $\HomoI{\lambda_{3}} \in \Sn$ by the definition of the homomorphic mapping of $\textsc{t}$ and $\textsc{t}^{'}$.
Therefore, $a \ast b \in \Sn$.


\SubProblem{Associativity: $\quad\AssociativityQuantification{\Sn}\quad\quad\AssociativityProperty{\ast}$}

Similarly to closure, associativity of $\left(\ast\right)$ can be easily derived from the associativity of function composition and the homomorphism $\textsc{t}$.

\begin{align*}
a \ast (b \ast c)~&=~\HomoI{\Homo{a \ast (b \ast c)}} \\
&=~\HomoI{\Homo{a} \circ \Homo{b \ast c}} \\
&=~\HomoI{\Homo{a} \circ (\Homo{b} \circ \Homo{c})} \\
&=~\HomoI{(\Homo{a} \circ \Homo{b}) \circ \Homo{c}} \\
&=~\HomoI{(\Homo{a \ast b}) \circ \Homo{c}} \\
&=~\HomoI{\Homo{(a \ast b) \ast c}} \\
&=~(a \ast b) \ast c \\
\end{align*}


\SubProblem{Identity: $\quad\IdentityQuantification{\Sn}\quad\quad\IdentityProperty{\ast}$}

Let $e = \bigl[0 \mapsto 0,\, 1 \mapsto 1,\, \hdots \,,\, n-1 \mapsto n-1,\, n \mapsto n \bigr]$

Observe the following equalities:

\begin{align*}
e \ast a~&=~\bigl[0 \mapsto 0,\, 1 \mapsto 1,\, \hdots \,,\, n \mapsto n \bigr] \ast \bigl[0 \mapsto x_0,\, 1 \mapsto x_1,\, \hdots \,,\, n \mapsto x_n \bigr] \\
&=~\bigl[(0 \mapsto 0) \circ (0 \mapsto x_0),\, (1 \mapsto 1) \circ (1 \mapsto x_1),\,  \hdots \,,\, (n \mapsto n) \circ (n \mapsto x_n) \bigr] \\
&=~\bigl[0 \mapsto x_0,\, 1 \mapsto x_1,\, \hdots \,,\, n \mapsto x_n \bigr] \\
&=~a \\
&=~\bigl[0 \mapsto x_0,\, 1 \mapsto x_1,\, \hdots \,,\, n \mapsto x_n \bigr] \\
&=~\bigl[(0 \mapsto x_0) \circ (0 \mapsto 0),\, (1 \mapsto x_1) \circ (1 \mapsto 1),\, \hdots \,,\, (n \mapsto x_n) \circ (n \mapsto n) \bigr] \\
&=~\bigl[0 \mapsto x_0,\, 1 \mapsto x_1,\, \hdots \,,\, n \mapsto x_n \bigr] \ast \bigl[0 \mapsto 0,\, 1 \mapsto 1,\, \hdots \,,\, n \mapsto n \bigr] \\
&=~a \ast e
\end{align*}


\SubProblem{Inverse: $\quad\InverseQuantification{\Sn}\quad\quad\InverseProperty{\ast}$}

For each $a = \bigl[0 \mapsto x_0,\, 1 \mapsto x_1,\, \hdots \,,\, n \mapsto x_n \bigr] \in \Sn$ we can construct its inverse $a^{'}$ in the following form:

\[
a^{'} = \bigl[x_0 \mapsto 0,\, x_1 \mapsto 1,\, \hdots \,,\, x_n \mapsto n \bigr]
\]

Observe the following equalities:

\begin{align*}
a \ast a^{'}~&=~\bigl[0 \mapsto x_0,\, 1 \mapsto x_1,\, \hdots \,,\, n \mapsto x_n \bigr] \ast \bigl[x_0 \mapsto 0,\, x_1 \mapsto 1,\, \hdots \,,\, x_n \mapsto n \bigr] \\
&=~\bigl[(0 \mapsto x_0) \circ (x_0 \mapsto 0),\, (1 \mapsto x_1) \circ (x_1 \mapsto 1),\, \hdots \,,\, (n \mapsto x_n) \circ (x_n \mapsto n) \bigr] \\
&=~\bigl[0 \mapsto 0,\, 1 \mapsto 1,\, \hdots \,,\, n \mapsto n \bigr] \\
&=~e \\
&~=\bigl[0 \mapsto 0,\, 1 \mapsto 1,\, \hdots \,,\, n \mapsto n \bigr] \\
&=~\bigl[(x_0 \mapsto 0) \circ (0 \mapsto x_0),\, (x_1 \mapsto 1) \circ (1 \mapsto x_1),\, \hdots \,,\, (x_n \mapsto n) \circ (n \mapsto x_n) \bigr] \\
&=~\bigl[0 \mapsto x_0,\, 1 \mapsto x_1,\, \hdots \,,\, n \mapsto x_n \bigr] \ast \bigl[x_0 \mapsto 0,\, x_1 \mapsto 1,\, \hdots \,,\, x_n \mapsto n \bigr] \\
&=~a^{'} \ast a
\end{align*}

\Problem{%
Show that $\mathcal{P}(x) = x^4 + x^3 + x^2 + x + 1$ is an irreducible polynomial for $\mathbb{F}_{2}\left[x\right]$.
}%

Use Sturm's theorem to count the number of roots of $\mathcal{P}(x)$ by constructing the Sturm sequence for $\mathcal{P}(x)$. The number of roots will equal to the difference in number of sign changes when evaluating the Sturm sequence for $\mathcal{P}(x)$ at $-\infty$ and $+\infty$.\\

Strum sequence for a polynomial $\textsc{p}$ has the form:
\begingroup
\setlength{\tabcolsep}{10pt} % Default value: 6pt
\renewcommand{\arraystretch}{1.5} % Default value: 1
\[
\begin{NiceArray}{ll}%
[margin,nullify-dots]
\textsc{p}_{0}(x) =& \textsc{p}(x) \\
\textsc{p}_{1}(x) =& \textsc{p}_{0}^{'}(x) \\
\textsc{p}_{i}(x) =& -1 *\text{rem}\left(\dfrac{\textsc{p}_{(i-2)}(x)}{\textsc{p}_{(i-1)}(x)}\right) \end{NiceArray}
\]
\endgroup



Strum sequence of $\mathcal{P}(x)$:
\begingroup
\setlength{\tabcolsep}{10pt} % Default value: 6pt
\renewcommand{\arraystretch}{2} % Default value: 1
\[
\begin{NiceArray}{lrr}
p_{0}(x) =& \mathcal{P}(x) =& x^4 +  x^3 +  x^2 +  x + 1 \\
p_{1}(x) =& p_{0}^{'}(x)   =&       4x^3 + 3x^2 + 2x + 1 \\
p_{2}(x) =& -1 *\text{rem}\left(\dfrac{p_{0}(x)}{p_{1}(x)}\right) =&  \dfrac{-5}{16}x^2 - \dfrac{5}{8}x - \dfrac{15}{16} \\
p_{3}(x) =& -1 *\text{rem}\left(\dfrac{p_{1}(x)}{p_{2}(x)}\right) =& -16
\end{NiceArray}
\]
\endgroup


Sign sequences derived from the Strum sequence of $\mathcal{P}(x)$ evaluated at $-\infty$ and $+\infty$:
\[
\left\{\; p_{0}(-\infty),\,  p_{1}(-\infty),\, p_{2}(-\infty),\, p_{3}(-\infty) \;\right\} = \left\{\; +,\, -,\, -,\, - \;\right\}
\]
\[
\left\{\; p_{0}(+\infty),\,  p_{1}(+\infty),\, p_{2}(+\infty),\, p_{3}(+\infty) \;\right\} = \left\{\; +,\, +,\, -,\, - \;\right\}
\]

Variation in sign $V(-\infty) = 1$ when evaluating the sequence at $-\infty$, notably between $p_{0}\left(-\infty\right)$ and $p_{1}\left(-\infty\right)$.

Variation in sign $V(+\infty) = 1$ when evaluating the sequence at $+\infty$, notably between $p_{1}\left(-\infty\right)$ and $p_{2}\left(-\infty\right)$.

The difference in sign variation is $V(-\infty) - V(+\infty) = 1 - 1 = 0$.
Hence there are no real-valued roots of $\mathcal{P}(x)$ in the interval $(-\infty, +\infty]$.
It also follows immediately that there are no factors of $\mathcal{P}(x)$ in $\mathbb{R}\left[x\right]$.
Note that $\{0,1\} \subset \mathbb{R}$ and that the field $\mathbb{F}_{2}\left[x\right]$ is a subspace of $\mathbb{R}\left[x\right]$.
Since the field $\mathbb{F}_{2}\left[x\right]$ is a more constrained domain than $\mathbb{R}\left[x\right]$, this result is stronger actually than what was required!
Therefore $\mathcal{P}(x)$ is an irreducible polynomial for $\mathbb{F}_{2}\left[x\right]$ \emph{and} $\mathbb{R}\left[x\right]$.


\end{document}
