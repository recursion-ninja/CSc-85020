%This is my super simple Real Analysis Homework template

\documentclass{AssignmentCUNY}
\usepackage[utf8]{inputenc}
\usepackage{enumitem}
\usepackage{amsmath}
\usepackage{mathrsfs} % https://www.ctan.org/pkg/mathrsfs
\usepackage[]{amsthm} %lets us use \begin{proof}
\usepackage[]{amssymb} %gives us the character \varnothing
\usepackage{nicematrix}

\setlength{\abovedisplayskip}{1pt}
\setlength{\belowdisplayskip}{1pt}
\setlength{\abovedisplayshortskip}{1pt}
\setlength{\belowdisplayshortskip}{1pt}


\renewcommand{\thefootnote}{\fnsymbol{footnote}}


\title{CSc 85020 - Coding Theory\\ Assignment 1}
\author{Alex Washburn}
\date\today
%This information doesn't actually show up on your document unless you use the maketitle command below

\AssignmentNumber{1}%
\CourseName{Coding Theory}
\CourseNumber{85020}
\StudentName{Alex Washburn}

\begin{document}
\CoverPage
\Problem{%
Find the generator matrix of a $r$-way repetition code.%
}%

Suppose we have messages of length $k$, and wish to use a $r$-way repetition code to produce codewords of length $k * r$.
The constructed generator matrix $\mathbf{G}$ will have dimensions $\langle\, k \times (k*r) \,\rangle$. We can construct $\mathbf{G}$ by concatenating $r$ identity matrices $I_{k}$. In general, the construction\footnote[2]{The diagonal $1$ values of the identity matrices are in bold face for clarity.} of $\mathbf{G}$ will take the following form:

\newcommand{\CodeWord}[1]{\ensuremath{\langle\,#1\,\rangle}}
\newcommand{\CodeSchema}[4]{\ensuremath{\boldsymbol{\bigl(\bigr.}\,#1,~#2,~#3\,\boldsymbol{\bigl.\bigr)}_{#4}}}
\newcommand{\VD}{\Vdots}
\newcommand{\R}[1]{\rotate~#1}
\newcommand{\C}[1]{#1\hspace*{1mm}}
\newcommand{\BO}{\ensuremath{\mathbf{1}}}
\newcommand{\slim}{\mkern-8mu}
\setcounter{MaxMatrixCols}{32}
\NiceMatrixOptions{code-for-first-row = \color{darkgray},
	code-for-first-col = \color{darkgray},
	code-for-last-row = \color{green},
	code-for-last-col = \color{magenta},
	cell-space-limits = 1pt
}

\[\mathbf{G}\quad=\quad\begin{pNiceMatrix}%
	[margin,name=RepetitionCode,first-row,first-col,nullify-dots]
	 ~  & \R 1& \R 2& \R 3& \Cdots & \R k&\R k+1&\R k+2&\R k+3& \Cdots &\R k*2& \overbrace{\hdots\hdots}^{\rotatebox{90}{\ensuremath{(k-3)*r}}} &\R k*r-(k-1)&\R k*r-(k-2)&\R k*r-(k-3)& \Cdots &\R k*r\\
	 1 & \BO &  0  &  0  & \Cdots &  0  & \BO  &  0   &  0   & \Cdots &  0   &            \hdots\hdots          & \BO        &  0         &  0         & \Cdots &  0   \\
	 2  &  0  & \BO &  0  & \Cdots &  0  &  0   & \BO  &  0   & \Cdots &  0   &            \hdots\hdots          &  0         & \BO        &  0         & \Cdots &  0   \\
	 3  &  0  &  0  & \BO & \Cdots &  0  &  0   &  0   & \BO  & \Cdots &  0   &            \hdots\hdots          &  0         &  0         & \BO        & \Cdots &  0   \\
	\VD & \VD & \VD & \VD & \Ddots & \VD & \VD  & \VD  & \VD  & \Ddots & \VD  &            \hdots\hdots          & \VD        & \VD        & \VD        & \Ddots & \VD  \\
	 k  &  0  &  0  &  0  & \Cdots & \BO &  0   &  0   &  0   & \Cdots & \BO  &            \hdots\hdots          &  0         &  0         &  0         & \Cdots & \BO  \\
\end{pNiceMatrix}\]


\Problem{%
Write the parity check matrix of the Hamming code with $r=4$.%
}%

All Hamming codes take the form $\CodeSchema{2^{r} - 1}{2^{r} - 1 - r}{3}{2}$.
The Hamming code with $r=4$ has the form:
\[ \CodeSchema{2^{4} - 1}{2^{4} - 1 - 4}{3}{2} = \CodeSchema{15}{11}{3}{2} \]

All Hamming codes are linear codes and also binary codes, therefore:

\begin{enumerate}

	\item Both the generator matrix $\mathbf{G}$ and parity check matrix $\mathbf{H}$ exist.

	\item $\mathbf{G}$ and $\mathbf{H}$ have dimensions $\langle\, k \times n \,\rangle = \langle\, 11 \times 15 \,\rangle$ and $\langle\, (n - k) \times n \,\rangle = \langle\, 4 \times 15 \,\rangle$, respectively.
	
	\item $\mathbf{G}$ and $\mathbf{H}$ have standard forms $\left[\,I_k,\:A\,\right]$ and $\left[\,A^{T},\:I_{n-k}\,\right]$, respectively.

\end{enumerate}

Hence we can construct the parity check matrix $\mathbf{H}$ (and $\mathbf{G}$) for the Hamming code with $r=4$ in standard form:

\[
\begin{array}{rcrllrcr}%
\mathbf{G}&\slim=\slim&[&\mkern-14muI_k,&\mkern-12muA\mkern-14mu&]&\slim=&\slim\begin{pNiceMatrix}%
	[margin,name=HammingGenerator,first-row,first-col,nullify-dots]
	~     &\R{1}&\R{2}&\R{3}&\R{4}&\R{5}&\R{6}&\R{7}&\R{8}&\R{9}&\R{10}&\R{11}&\R{12}&\R{13}&\R{14}&\R{15} \\
	\C{ 1}& \BO &  0  &  0  &  0  &  0  &  0  &  0  &  0  &  0  &  0   &  0   &  1   &  1   &  1    &  1   \\
	\C{ 2}&  0  & \BO &  0  &  0  &  0  &  0  &  0  &  0  &  0  &  0   &  0   &  1   &  1   &  1    &  0   \\
	\C{ 3}&  0  &  0  & \BO &  0  &  0  &  0  &  0  &  0  &  0  &  0   &  0   &  1   &  1   &  0    &  1   \\
	\C{ 4}&  0  &  0  &  0  & \BO &  0  &  0  &  0  &  0  &  0  &  0   &  0   &  1   &  1   &  0    &  0   \\
	\C{ 5}&  0  &  0  &  0  &  0  & \BO &  0  &  0  &  0  &  0  &  0   &  0   &  1   &  0   &  1    &  1   \\
	\C{ 6}&  0  &  0  &  0  &  0  &  0  & \BO &  0  &  0  &  0  &  0   &  0   &  1   &  0   &  1    &  0   \\
	\C{ 7}&  0  &  0  &  0  &  0  &  0  &  0  & \BO &  0  &  0  &  0   &  0   &  1   &  0   &  0    &  1   \\
	\C{ 8}&  0  &  0  &  0  &  0  &  0  &  0  &  0  & \BO &  0  &  0   &  0   &  0   &  1   &  1    &  1   \\
	\C{ 9}&  0  &  0  &  0  &  0  &  0  &  0  &  0  &  0  & \BO &  0   &  0   &  0   &  1   &  1    &  0   \\
	\C{10}&  0  &  0  &  0  &  0  &  0  &  0  &  0  &  0  &  0  & \BO  &  0   &  0   &  1   &  0    &  1   \\
	\C{11}&  0  &  0  &  0  &  0  &  0  &  0  &  0  &  0  &  0  &  0   & \BO  &  0   &  0   &  1    &  1   \\
\end{pNiceMatrix} \\
\hspace*{4mm}\\
\mathbf{H}&\slim=\slim&[&\mkern-14muA^{T},&\mkern-12muI_{n-k}\mkern-14mu&]&\slim=&\slim\begin{pNiceMatrix}%
	[margin,name=HammingParityCheck,first-row,first-col,nullify-dots]
	~     &\R{1}&\R{2}&\R{3}&\R{4}&\R{5}&\R{6}&\R{7}&\R{8}&\R{9}&\R{10}&\R{11}&\R{12}&\R{13}&\R{14}&\R{15} \\
        \C{1}&  1  &  1  &  1  &  1  &  1  &  1  &  1  &  0  &  0  &  0   &  0   & \BO  &  0   &  0    &  0    \\
        \C{2}&  1  &  1  &  1  &  1  &  0  &  0  &  0  &  1  &  1  &  1   &  0   &  0   & \BO  &  0    &  0    \\
        \C{3}&  1  &  1  &  0  &  0  &  1  &  1  &  0  &  1  &  1  &  0   &  1   &  0   &  0   & \BO   &  0    \\
        \C{4}&  1  &  0  &  1  &  0  &  1  &  0  &  1  &  1  &  0  &  1   &  1   &  0   &  0   &  0    & \BO   \\
\end{pNiceMatrix}

\end{array}\]%


\Problem{%
Construct a standard array for the binary linear code with generator matrix $G = \left[\begin{smallmatrix}
	1 & 1 & 0 & 1 & 0\\
	0 & 1 & 0 & 1 & 0
\end{smallmatrix}\right]$. Then decode the words \texttt{11111} and \texttt{10000}.%
}%

We can enumerate all codewords $\vec{c} \in \mathbf{C}$ by multiplying each message $\vec{m} \in \mathbf{M} = \mathbb{Z}_{2}^{2}$ by the generator matrix $\mathbf{G}$:

\begingroup
\setlength{\tabcolsep}{10pt} % Default value: 6pt
\renewcommand{\arraystretch}{1.5} % Default value: 1
\begin{center}
\begin{tabular}{ c | c }
	$\vec{m} \in \mathbf{M}$ & $\vec{c} = \vec{m}*\mathbf{G}$ \\
	\hline
	\CodeWord{ 0, 0 } & \CodeWord{ 0, 0, 0, 0, 0 } \\
	\CodeWord{ 0, 1 } & \CodeWord{ 0, 1, 0, 1, 0 } \\
	\CodeWord{ 1, 0 } & \CodeWord{ 1, 1, 0, 1, 0 } \\
	\CodeWord{ 1, 1 } & \CodeWord{ 1, 0, 0, 0, 0 } \\
\end{tabular}
\end{center}
\endgroup


\[ \mathbf{C} = \Bigl\{\;\; \vec{c} = \vec{m}*H \;\;|\;\; \vec{m} \in \mathbf{M} \;\;\Bigr\} = \Bigl\{\;\; \CodeWord{ 0, 0, 0, 0, 0 }, \CodeWord{ 0, 1, 0, 1, 0 }, \CodeWord{ 1, 1, 0, 1, 0 }, \CodeWord{ 1, 0, 0, 0, 0 } \;\;\Bigr\}\]

Next we can construct the standard array with each column representing the coset of a codeword $\vec{c} \in \mathbf{C}$:

\begingroup
\setlength{\tabcolsep}{10pt} % Default value: 6pt
\renewcommand{\arraystretch}{1.5} % Default value: 1
\[
\NiceMatrixOptions{cell-space-limits=1pt}
\begin{pNiceArray}[t]{c|cccc}[baseline=line-3]%
\text{Syndrome} & x + \vec{c}_{0,0} & x + \vec{c}_{0,1} & x + \vec{c}_{1,0} & x + \vec{c}_{1,1} \\
\hline
\C{000} & \CodeWord{ 0, 0, 0, 0, 0 } & \CodeWord{ 0, 1, 0, 1, 0 } & \CodeWord{ 1, 1, 0, 1, 0 } & \CodeWord{ 1, 0, 0, 0, 0 } \\
\hline
\C{001} & \CodeWord{ 0, 0, 0, 1, 0 } & \CodeWord{ 0, 1, 0, 0, 0 } & \CodeWord { 1 0 0 1 0 } & \CodeWord{ 1 1 0 0 0 } \\
\C{010} & \CodeWord{ 0, 0, 1, 0, 0 } & \CodeWord{ 0, 1, 1, 1, 0 } & \CodeWord { 1 0 1 0 0 } & \CodeWord{ 1 1 1 1 0 } \\
\C{011} & \CodeWord{ 0, 0, 1, 1, 0 } & \CodeWord{ 0, 1, 1, 0, 0 } & \CodeWord { 1 0 1 1 0 } & \CodeWord{ 1 1 1 0 0 } \\
\C{100} & \CodeWord{ 0, 0, 0, 0, 1 } & \CodeWord{ 0, 1, 0, 1, 1 } & \CodeWord { 1 0 0 0 1 } & \CodeWord{ 1 1 0 1 1 } \\
\C{101} & \CodeWord{ 0, 0, 0, 1, 1 } & \CodeWord{ 0, 1, 0, 0, 1 } & \CodeWord { 1 0 0 1 1 } & \CodeWord{ 1 1 0 0 1 } \\
\C{110} & \CodeWord{ 0, 0, 1, 0, 1 } & \CodeWord{ 0, 1, 1, 1, 1 } & \CodeWord { 1 0 1 0 1 } & \CodeWord{ 1 1 1 1 1 } \\
\C{111} & \CodeWord{ 0, 0, 1, 1, 1 } & \CodeWord{ 0, 1, 1, 0, 1 } & \CodeWord { 1 0 1 1 1 } & \CodeWord{ 1 1 1 0 1 } \\
\end{pNiceArray}
\]
\endgroup

With the standard array assembled, we can decode both the words $\CodeWord{1, 1, 1, 1, 1}$ as well as the codeword $\CodeWord{1, 0, 0, 0, 0}$ to message the (coincidentally) same message $\CodeWord{ 1, 1 }$.

%Notes on block codes:
%
%\begin{align}
%\text{Notation:      }& \left( n, k, d \right)_{q}\\
%\text{Alphabet:      }& \Sigma\\
%\text{Codewords:     }& \mathbf{C} \subset \Sigma^{n}\\
%\text{Messages:      }& \mathbf{M} \subset \Sigma^{k}\\
%\text{Codeword Size: }& n \\
%\text{Message  Size: }& k \\
%\text{Distance:      }& d = \min\left\{ \Delta(x,y) : x,y \in \mathbf{C} x \not= y \right\}\\
%\text{Symbols:       }& q = \left\lvert\Sigma\right\rvert
%\end{align}


\end{document}
